\chapter{Implementation}



Topology wise, nodes are randomly place in a square area which is proportional to the number of nodes. The equation to calculate the size of the square is as follows:

(Insert the equation here).

Every node's coordinates are used to calculate neighbors list for each node. In practice, the global access of this information is usually not easy to obtain. Thus, Hello packets are used to compile neighbors list for each node.

Because nodes are randomly placed in a square area, with fixed WiFi range there could be a case that each node has at least one neighbor but the network is separated into two subnets unconnected. [twoSepNet.png]

Therefore, I used an algorithm that uses depth-first search algorithm to determine whether all nodes get visited during the recursive search. If the algorithm sucessfully traverse all nodes, it is considered a complete graph which means the network is connected. If the algorithm yield a failure, this trial will be rejected and the simulation will move on to next trial.

In order to collect benchmark of adaptive fanout gossip protocol, one extra node is place in the area. Its functions are generate new packets, collect other nodes' received time of every packets, and store new packets generate time. A trimmed version of adaptive fanout gossip protocol is used to generate new packets. An UDP server application is installed on the node to collect other nodes' received time of every packets.

Adaptive fanout gossip protocol is installed on all other nodes in the network. Besides that, a UDP client application is also installed on them to send received time of every packet. The simulation stops when any one node's energy is depleted. After that, all the data collected is processed to generate our performance metrics. 


\section{A Section}



\subsection{A Subsection}



\section{Another Section}

