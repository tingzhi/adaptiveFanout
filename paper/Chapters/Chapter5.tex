\chapter{PREVIOUS WORK}

\subsection{Virtual Sensor Networks}
The ongoing technological progress further and further improves the computation, connectivity and sensing capabilities of various devices, sometimes mobile ones. \cite{Jayasumana} This enables a huge variety of opportunities in sensor networks. For example, devices in a sensor network could be assigned tasks based on their constraints in computation, power usage or networking potential. In contrast to dedicated sensor networks, where the participating nodes serve a single application, Virtual Sensor Networks (VSN) take advantage of the node’s technological progress. When a VSN is formed on top of a Wireless Sensor Network, only a subset of all available nodes is part in the VSN. Furthermore, several VSNs can exist simultaneously in on Wireless Sensor Network. \cite{Jayasumana} That is, one subset of the nodes forms a VSN and relies on the remaining nodes to communicate between its nodes. In some cases, physical nodes of one VSN even could be completely cut off from communication due to their spatial distribution and must rely on the other nodes. Usually the different VSNs pursue completely unrelated sensing tasks and the nodes in each VSN behave like they are on their independent Sensor Network. Figure~\ref{vsnfig} based on \cite{Jayasumana} depicts a visualization of two VSNs formed on top of an Wireless Sensor Network. This logical separation helps to simplify the implementation of applications significantly. \cite{Jayasumana} Further advantages of VSNs are enhanced performance and better scalability.

The development of algorithms and protocols to support the grouping of VSNs on top of Sensor Networks, is still an ongoing research topic. Those need to consider how the available time and frequencies should be fairly distributed for intra network communication. Moreover, it should be possible for nodes to change their membership in VSNs.

%\begin{figure}
%	\centering
%	\includegraphics[width=3in]{figs/VirtualSensorNetwork.pdf}
%	\caption{Broadcast path from a node (S) in VSN2.}
%	\label{vsnfig}
%\end{figure}


\subsection{Virtual Networks on Top of the Internet}
It is important to realize that the Internet, due to so many different participants with sometimes opposing interests, is hard to modify and only possible small and slow steps, if at all. Therefore, Virtual Networks are often the only way to realize innovation. To implement a Virtual Network using the existing Internet, several things need to be considered. First, the characteristics of the networking technology determine the attributes of the Virtual Network. For instance, a wired network yields a more scalable and bandwidth flexible Virtual Network than a wireless network would do. \cite{Chowdhury} Second, the layer of virtualization (referring to the OSI layer model) impacts the flexibility of the Virtual Network. That is, the lower the layer of virtualization, the more flexibility will be possible. Specifically, so-called overlay networks, mostly realized in the application layer, are limited in their ability to support fundamentally different architectures. \cite{Chowdhury} Moreover, virtualization on top of IP is fixed to the network layer protocol and cannot deploy IP independent mechanisms.  \cite{Chowdhury} Lastly, an important consideration in the non-comprehensive list is also about security and privacy in virtual networks. Thus, attack vectors such as denial-of-service or distributed denial-of-service against the underlying physical network will have impact on all simultaneously virtualized networks.

\subsection{Virtualization Algorithm}
Though, it is possible to form a VSN of mobile IoT devices by having access to all relevant data such as availability, sensor capabilities or sensor mobility, a more efficient solution is to assume the managing cloud agent does not have full knowledge of every sensors’ properties. \cite{Sherif} The cloud agent even may not be connected to all nodes but only to a subgroup of them. The presented algorithm also takes into account mobility of the devices which sometimes leads to nodes being unavailable for some time. \cite{Sherif}
This virtualization algorithm will search and select appropriate sensors from the whole network to form the virtual network which then executes the sensing task.




Lorem ipsum dolor sit amet, consectetur adipiscing elit. Vivamus at pulvinar nisi. Phasellus hendrerit, diam placerat interdum iaculis, mauris justo cursus risus, in viverra purus eros at ligula. Ut metus justo, consequat a tristique posuere, laoreet nec nibh. Etiam et scelerisque mauris. Phasellus vel massa magna. Ut non neque id tortor pharetra bibendum vitae sit amet nisi. Duis nec quam quam, sed euismod justo. Pellentesque eu tellus vitae ante tempus malesuada. Nunc accumsan, quam in congue consequat, lectus lectus dapibus erat, id aliquet urna neque at massa. Nulla facilisi. Morbi ullamcorper eleifend posuere. Donec libero leo, faucibus nec bibendum at, mattis et urna. Proin consectetur, nunc ut imperdiet lobortis, magna neque tincidunt lectus, id iaculis nisi justo id nibh. Pellentesque vel sem in erat vulputate faucibus molestie ut lorem.

\section{A Section}

Quisque tristique urna in lorem laoreet at laoreet quam congue. Donec dolor turpis, blandit non imperdiet aliquet, blandit et felis. In lorem nisi, pretium sit amet vestibulum sed, tempus et sem. Proin non ante turpis. Nulla imperdiet fringilla convallis. Vivamus vel bibendum nisl. Pellentesque justo lectus, molestie vel luctus sed, lobortis in libero. Nulla facilisi. Aliquam erat volutpat. Suspendisse vitae nunc nunc. Sed aliquet est suscipit sapien rhoncus non adipiscing nibh consequat. Aliquam metus urna, faucibus eu vulputate non, luctus eu justo.

\subsection{A Subsection}

Donec urna leo, vulputate vitae porta eu, vehicula blandit libero. Phasellus eget massa et leo condimentum mollis. Nullam molestie, justo at pellentesque vulputate, sapien velit ornare diam, nec gravida lacus augue non diam. Integer mattis lacus id libero ultrices sit amet mollis neque molestie. Integer ut leo eget mi volutpat congue. Vivamus sodales, turpis id venenatis placerat, tellus purus adipiscing magna, eu aliquam nibh dolor id nibh. Pellentesque habitant morbi tristique senectus et netus et malesuada fames ac turpis egestas. Sed cursus convallis quam nec vehicula. Sed vulputate neque eget odio fringilla ac sodales urna feugiat.

\section{Another Section}

Phasellus nisi quam, volutpat non ullamcorper eget, congue fringilla leo. Cras et erat et nibh placerat commodo id ornare est. Nulla facilisi. Aenean pulvinar scelerisque eros eget interdum. Nunc pulvinar magna ut felis varius in hendrerit dolor accumsan. Nunc pellentesque magna quis magna bibendum non laoreet erat tincidunt. Nulla facilisi.

Duis eget massa sem, gravida interdum ipsum. Nulla nunc nisl, hendrerit sit amet commodo vel, varius id tellus. Lorem ipsum dolor sit amet, consectetur adipiscing elit. Nunc ac dolor est. Suspendisse ultrices tincidunt metus eget accumsan. Nullam facilisis, justo vitae convallis sollicitudin, eros augue malesuada metus, nec sagittis diam nibh ut sapien. Duis blandit lectus vitae lorem aliquam nec euismod nisi volutpat. Vestibulum ornare dictum tortor, at faucibus justo tempor non. Nulla facilisi. Cras non massa nunc, eget euismod purus. Nunc metus ipsum, euismod a consectetur vel, hendrerit nec nunc.