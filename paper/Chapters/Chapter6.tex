\chapter{Conclusions and Future Work}
\label{Chapter6}
\lhead{Chapter 6. \emph{Conclusions and Future Work}} % Write in your own chapter title to set the page header

In this thesis, we introduced research progress in gossip techniques for wireless broadcasting in recently years. We pointed out that despite all these efforts dedicated into reduing gossip broadcasting protocol overhead, very little research has focused on energy efficiency and network lifetime. Those aspect used to be not very important when designing new broadcasting protocols because nodes usually have stable power supplys. However, with the emergence of Internet of Things (IoT) devices, broadcasting protocols that take energy consumption  into account and optimizing it will be favored over those that do not. Based on our observation of the tradeoff between battery life and broadcasting time regarding \emph{fanout} parameter, we proposed a new energy-aware gossip broadcasting protocol that could balance between network lifetime and broadcasting time. In order to evaluate the performance of our proposed approach, we designed several metrics and we developed the protocol in open-source software ns-3. Simulation results showed that comparing to constant $f=5$ setting, our adaptive approach significantly extended network lifetime while only performed slightly slower in term of message broadcasting time. We suspect the casue for marginal performance improvements for $f=10$ setting comparing to $f=5$ setting is the average node's degree. In other words, very little performance boost can be observed when \emph{fanout} is set beyond average node's degree since a node simply cannot reach out to 10 neighbors when it only has 5 neighbors. As we discussed earlier, $f=1$ setting has the worst message broadcasting time. But on the flip side, it would result in a longest network lifetime which could be desirable for some applications. 

In conclusion, our proposed energy-aware adaptive gossip broadcasting protocol can leverage the advantages of low \emph{fanout} setting and high \emph{fanout} setting. Thus, we can extend network lifetime while still perform as good as high \emph{fanout} setting in term of broadcasting time. Constant $f=1$ setting is recommended for networks that consists of nodes with strict energy constrain. 

For future work, we would like to use multicast instead of multiple unicast for each node to send the new message. The reason is that if we assume $X J$ is the amount of energy used to transmit a packet for a sender, and $f=5$, one multicast will only consum $X J$ while five unicast will consum $5X J$. Another interesting scenerio would be to set very different initial energy but same battery capacity for each node. Thus each node would be operating at different \emph{fanout} setting from the begining due to different remaining energy fraction. I believe this would better capture the advantage of our proposed appraoch over constant \emph{fanout} setting.