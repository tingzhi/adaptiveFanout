\chapter{Conclusions and Future Work}

(from 563 paper).

We introduced the topics of Internet of things and sensor networks. The opportunities resulting by virtualization of those sensor networks have been elaborated. A distributed algorithm to disseminate message across a ad-hoc network has been implemented and evaluated. We proposed several performance metrics to evaluate this our proposed approach.

We have implemented the gossip protocol in a wireless ad-hoc network environment and run the simulation in ns-3. 

(!!!!update)The result from ns-3 shows that the average number of data packets sent per node almost remained constant when the network scale grows. Lastly, our results imply that the time until all nodes have received the data will grow linearly in a wireless ad-hoc network environment. Unfortunately, we could not verify that the time complexity of the algorithm is $O(\log n)$ as outlined in \cite{gossip}.

In the future work, I would like to investigate how request interval time and gossip interval time would impact the performance of gossip protocol. In my estimation, if we increase the gossip interval time and decrease the request interval time, average number of hops per node would increase. Spread time would potentially decrease and average data packets per node would not be affected.

Another direction would be implementing smart routing protocol. For example, a node can improve routing (or decrease the number of hops) by notifying neighbors if there exists a shorter route to the owner of a message. To accomplish this, the program need to keep track of the number of hops and then compare if the owner of a received message is one of their immediate neighbors. In case this is true and they will notify the source of the message, that the \textit{true} (or ideal) number of hops would be 1. Another option for further work, would be writing the code to generate more continous topology files in term of number of nodes so that we could better investigate the characteristic of gossip protocol.

Due to time limitations we could not address the huge spectrum of possible analysis. Power limitations and mobility of the nodes should be included in such a scenario. Thus, the list of neighbors for each node changes over time and a topology file is only need for initialization.

\section{A Section}

\subsection{A Subsection}
