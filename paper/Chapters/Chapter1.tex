\chapter{Introduction}
\label{Chapter1}
\lhead{Chapter 1. \emph{Introduction}} % Write in your own chapter title to set the page header

Along with the development of the information technology, the price of the broadband connectivity becomes affordable. Devices are trending to be smaller and more powerful. People start to explore ways to connect devices to the network for better control and monitor the status of them. Devices such as smart phone, smart watch, smart thermostats, and radio-frequency identification (RFID) tags are able to connect to a network and these devices could communicate with each other. If these devices are connected to the Internet as well, we call them the \textit{Internet of Things}. There is no doubt that IoT is an innovative paradigm \cite{Atzori} because this idea combined the Internet with our everyday gadgets. Either from the perspective of private user or from the perspective of business user, there are infinite possible ways to exploit IoT. As of today, research in the area of IoT is emerging rapidly and many open questions remain to be answered.

Many application services that IoT devices can provide rely on a network broadcast protocol to disseminating information \cite{smart}. For example, IoT devices firmware update package can be broadcast among devices in a distributed manner. However, the main challenge here is that IoT devices are often resource limited meaning they have limited bandwidth and energy, and restricted by their mobility. Due to the characteristics of IoT devices, topology of physical network formed from IoT devices is often dynamic. Therefore, a scalable, robust, and fault-tolerant broadcast protocol is needed for these dynamic networks. Flooding is consider to be the simplest broadcast protocol but it is not suitable in this application. This is because flooding result in excessive overhead, media contention, and packets collision \cite{tseng2002broadcast} which would severely deplete devices' precious battery power. Gossip techniques instead offer a relatively simple, robust, fast and probabilistic approach. This technique is inspired by the form of gossip seen in social network.

Besides gossip techniques, several deterministic approaches have been proposed which aim to reduce overhead by shifting \msg ~forward responsibility to a subset of nodes in the network \cite{smart}. However, there are two main problems for these approaches. First, if any node in the subset fails, nodes that are depend on it will not be able to receive new \msgs \cite{smart}. Second, the energy on nodes in those subset will be depleted sooner than other nodes that are not in the subset \cite{smart}.

To properly define a variation of gossip technique, three main parameters need to specified. They are \emph{\pog}, \emph{Fanout}, and \emph{Message Live Time}. With gossiping, nodes in the network have to forward the \msg ~with probability $p_{gossip} \leq 1$ \cite{smart}. The idea is that a message can be broadcast successfully without every nodes' participation \cite{smart}. This approach can achieve a lower overhead because only a portion of the nodes participated in gossiping. However, the right $p_{gossip}$ can be difficult to choose because global topology information is needed. Furthermore, due to the dynamic topology nature of IoT devices formed network, an optimal $p_{gossip}$ can become sub-optimal over time \cite{smart}. 

In terms of energy consumption, several adaptive energy based probabilistic schemes have been proposed. Most of them focused on dynamically adjusting $p_{gossip}$ based on energy level related parameters. D. Nitnaware et. al. proposed an adaptive \gp ~based on node's energy level in \cite{nitnaware2009performance}. The authors defined an energy level threshold, when a node's energy level is above the threshold, it will gossip the \msg ~with a fixed $p_{gossip}$. When a node's energy is below the threshold, it will drop the \msg. For a special case where a node only has one neighbor, the \emph{\pog} is set to be 1 regardless of its energy level \cite{2015survey}. This approach adjusts \emph{\pog} in a very coarse manner since a node would only operate in one of two states: gossip with a fixed probability, or drop incoming new \msgs. In \cite{nitnaware2010energy}, node's remaining energy fraction is used directly as \emph{\pog}. Clearly, this is more fine-tuned than \cite{nitnaware2009performance}.

However, none of these efforts focused on another key gossip technique parameter: \emph{Fanout}. From our observation, for any given \emph{\pog}, higher \emph{Fanout} setting allows nodes to contact more neighbors each round thus can achieve faster message broadcast time. But higher \emph{Fanout} setting usually is associated with higher energy consumption. On the other hand, lower \emph{Fanout} setting conserves nodes' battery power but it takes longer to broadcast a \msg. Our aim in this thesis is to retain the benefit of high \emph{Fanout} setting (fast message broadcast time), and yet increase the lifetime of the network. Therefore, we proposed a gossip broadcast protocol that can dynamically adjust \emph{Fanout} parameter based on each gossip node's remaining energy level.

The rest of this thesis is organized as follows. We present related work in Chapter 2. Chapter 3 describes the classic \gp, our basic \pp ~\gp, and our proposed energy-aware adaptive fanout extension. Chapter 4 presents the implementation of our energy-aware \gp. We present performance evaluation in Chapter 5, and conclusion and future work in Chapter 6.

%The gossip protocol could be used to build routing table[?], perform multicast[?], or in this case, perform broadcast.

% gossip techniques can be used to design routing protocol, multicast, broadcast.

%Some proposed a event counter based scheme to combat this issue. The gist of that shceme is that if a node overheard the same messages $a$ times and $a > b$ were $b$ is the threshold, it would not gossip its latest message this time. Some even went a step further, they tried to identify the dependency among a node and its neighbors and dynamically adjust gossip probability based on collected information.

%As we are moving to an IoT and mobile devices dominated world, energy conservation become more and more important as to overall user experience or network survival time. 