\chapter{Introduction}
% 1. describe the problem
% start with a example
% 2. state your contribution



summarize what I have done so far.

trickle algorithm in a gist: 
The main idea of this paper is ....

The gossip protocol could be used to build routing table[?], perform multicast[?], or in this case, perform broadcast.

There is no doubt that the \emph{Internet of Things (IoT)} is an innovative paradigm, \cite{Atzori} which is gaining popularity in our modern society.  With the development of information technology, digital devices are getting smaller and yet more powerful. The basic ideal of IoT is that with "unique addressing schemes", various of \emph{things} or \emph{objects} such as smart phones, watches, thermostats, Radio-Frequency IDentification (RFID) tags, sensors are able to communicate, cooperate with each other to perform tasks \cite{Atzori}.  

%As of today, research in the area of IoT is emerging rapidly and many open questions remain to be answered.

Remote sensing is one of the promising services of IoT. With remote sensing, the users could retrieve collected data through the network instead of physically retrieving data. Remote sensing involves the search and selection of IoT devices to form a virtual sensor network. Afterwards, sensing task is sent to the virtual sensor network. The selected devices then perform sensing collaboratively and report the result back to the remote cloud agent. 

There is one step during remote sensing process that I am particularly interested in, which is sending tasks to the virtual sensor network. Ther are two aspects of this area. One question is that what kind of network would IoT devices form? The other question is that what kind of protocol is efficient and robust for multicasting the sensing task? As we know, IoT devices often have limited bandwidth and power. Their mobility further impacted the topology of the selected virtual sensor network. Due to the characteristics of IoT devices, ad-hoc network become the most suitable tool to model IoT devices formed virtual seonsor network. In this paper, we choose gossip protocol\cite{gossip} as our multicast protocol and evaluate its performance.


\section{A Section}

Quisque tristique urna in lorem laoreet at laoreet quam congue. Donec dolor turpis, blandit non imperdiet aliquet, blandit et felis. In lorem nisi, 

\subsection{A Subsection}

Donec urna leo, vulputate vitae porta eu, vehicula blandit libero. Phasellus eget massa et leo condimentum mollis. Nullam molestie, justo at 

\section{Another Section}

Phasellus nisi quam, volutpat non ullamcorper eget, congue fringilla leo. Cras et erat et nibh placerat commodo id ornare est. Nulla facilisi. Aenean 