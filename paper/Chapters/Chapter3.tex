\chapter{Energy-aware Gossip}

The objective of gossip protocol is to broadcast messages in an efficient way by mimicing social activities when people spread rumors in office. It works as follows: when a node received a new message, it then will send it to multiple nodes in the network. The number of nodes it contacted is defined as the \emph{fan-out} of gossip protocol. It is denoted as $f$. Each time when a node face the decision of sending a new message to another node, the probability of doing so is defined as $p_{gossip}$. In the rest of the thesis, I will refer to it as $p_g$. $p_g$ is usually between 0 and 1. Once a node received a new message, the number of times it will contact other nodes is defined as \emph{message live time}. It is denoted as $T_l$.



\section{Classcial Gossip}
For classic gossip protocol in wired network sceniro, \emph{probability of gossip} is set to be 1 and \emph{fan-out} is usually set to be 1 or 2. 
\emph{message live time} could vary depending on the requirement. 

\subsection{How it works}

\subsection{All Parameters}

\subsection{Different Approach}

\section{Proposed Adaptive Gossip}



\subsection{Intuition}



\section{Key Intuition of Our Approach}

In order to save each node's energy and potentially extend the lifespan of the whole network. My idea is to use each node's remaining energy fraction (much like the battery indicator on your phone) as the X for the function of one of gossip protocol's parameter — fanout. The more energy a node has, the bigger the fanout will be for it. Thus, compare to constant fanout protocol, which does not take nodes' energy into account, our apprach — adaptive fanout will dynamically adjust fanout based on a node's remaining energy.

Why not adjust gossip probability based on remaining energy fraction?

Why not adjust other gossip protocol parameters based on remaining energy fraction?



(from 563 paper).
\section{From gossip course paper}

To achieve the goals stated in section~\ref{sec:problem}, I took four crucial steps.

\begin{itemize}
	\item Extend the Internet Control Message Protocol (ICMP) to support transmitting three simple control messages needed for gossip protocol.
	\item Develop the gossip protocol application to be installed on network nodes.
	\item Set wireless ad-hoc network attributes and gossip protocol attributes, which has already be presented in section~\ref{sec:problem}.
	\item Import nodes connectivity information from topology files and export simulation results for performance evaluation.
\end{itemize}

\subsection{ICMP Extension}

ICMP stands for Internet Control Message Protocol. The most common use of ICMP is for error reporting~\cite{james}. A ICMP message contains two parts: 8-byte header and data section. The first 4 btyes of the header have fixed format. However, the last 4 bytes vary and depend on the type or code of the ICMP packet~\cite{forouzan}. The first and second byte of the header is the type field and code field respectively. And the third and fourth byte are checksum field. The format of the header is shown in table \ref{table:1}.

\begin{table}[h!]
	\centering
	\caption{ICMP Header Structure}
	\label{table:1}
	\begin{tabular}{|p{1 cm}|p{1 cm}|p{1 cm}|p{1 cm}|p{1 cm}|}
		\hline
		Octet & 0 & 1 & 2 & 3 \\
		\hline
		& Type & Code & 
		\multicolumn{2}{ |c| }{Checksum}  \\
		\hline
		Octet & 4 & 5 & 6 & 7 \\
		\hline
		& 
		\multicolumn{4}{|c|}{Rest of Header}  \\
		\hline
	\end{tabular}
\end{table} 

Table \ref{table:2} here presented some selected ICMP message types. 

\begin{table}[h]
	\centering
	\caption{Control Messages}
	\label{table:2}
	\begin{tabular}{|p{1.5cm}|p{0.8 cm}|p{4.5 cm}|}
		\hline
		Type & Code & Description \\                                                           
		\hline
		0  & 0   & Echo reply   \\ \hline
		8  &  0 & Echo request \\ 
		\hline
		9 & 0 & Router Advertisement \\
		\hline
		10	& 0	&	Router discovery/selection/solicitation \\
		\hline
		42 to 255    &   & Reserved    \\ 
		\hline
	\end{tabular}
\end{table}

Since type 42 to 255 are reserved for further development, I decided to extend ICMP by defining type 42, 43, and 44 to represent ackownlegement packet, request packet, and data packet respectively. The detail is shown in table \ref{table:3}.

\begin{table}[h]
	\centering
	\caption{Gossip Protocol Control Messages}
	\label{table:3}
	\begin{tabular}{|p{0.8cm}|p{0.5 cm}|p{3.5 cm}|}
		\hline
		Type & Code & Description \\                                                           
		\hline
		42  & 0   & Send Acknowledgment   \\ \hline
		43  &  0 & Send Request \\ 
		\hline
		44 & 0 & Send Data \\
		\hline
	\end{tabular}
\end{table}

Upon these new control message types extension, we could further develop gossip protocol in ns-3.

\subsection{Gossip Protocol}

Gossip protocol is a computer communication protocol which inspired by the social activity -- gossip. This protocol accomplishes to synchronize a message in a network that does not need real-time synchronization. It provides $O(\log n)$ time to synchronize the message to the network, where $n$ means the number of nodes in the network. There are three packet types for the message protocol: Data packet, ACK packet and request packet as mentioned above. The data packet contains the message from a certain source node that is expected to be disseminated to the whole network. The ACK packet is used to acknowledge to source node that destination node already received message before. The request packet is used for source node to request for the message from destination node. There are two states for each node, running and stop.

\begin{figure}
	\centering
	\begin{verbatim}[fontsize=\small]
	switch(state):
	case running:
	if message is not null:
	every 5 milliseconds:
	find a random neighbor R
	send data packet to R
	if receive a packet:
	if packet is ACK:
	state <- stop
	if packet is data:
	send ACK to packet source
	else:
	if receive a packet:
	if packet is a data packet:
	update the message
	every 5 seconds:
	find a random neighbor R
	send a request packet to R
	if receive a packet:
	if packet is a data packet:
	update the message to data
	
	case stop:
	if receive a packet:
	if packet is request:
	send data to the source node
	if packet is data:
	send ACK to the source node
	\end{verbatim}
	\caption{The pseudo code of the gossip algorithm.}
	\label{fig:pseudo}
\end{figure}

The pseudo code of gossip algorithm is given in figure~\ref{fig:pseudo}. All nodes in the network have same behavior and they are independent to each other. When user would like to spread a message x to the whole network, the protocol randomly assign the message x to a node and run this algorithm. For the discussion, let's assume node 1 got the initial message x. Node 1 start with running state, and node 1 has a message x, so it will find a random neighbor every 0.005s and send a data packet which contain message x to its random neighbor. If node 1 receive a ACK packet it will go to stop state, otherwise it will keep doing above steps. Other nodes which do not have the message x will wait for packet, if a node 1 send a data packet to the node 2 which do not have message x, node 2 will update the message x in node 2 and not return anything to node 1. If node 2 do not have message 2, it will send a request packet to a random neighbor every 5 second. If node 2 receives a data packet after sending the request packet, it will update its message to x. The nodes in stop state would always waiting for other node send control messages to it and it will send data packet or ACK packet back depends on what kind of packets it receives.

\subsection{Gathering Simulation Data}

To evaluate the performance of gossip protocol, we use several randomly generated topology files with the number of nodes as variable. Those topology files are derived from a random geometric graph network, which was created by uniformly and randomly placing nodes into a space and then connect nodes whose distance is smaller than some given radius.

\begin{figure}
	\centering
	\begin{verbatim}[fontsize=\small]
	#Nodes
	0
	1
	2
	#Edges
	(0, 1)
	(1, 2)
	\end{verbatim}
	\caption{A topology file of a linear topology with three nodes.}
	\label{fig:topsimple}
\end{figure}


Each topology file contains the number of nodes as well as all the edges, which represents the connections between nodes. As an example, the content of a simple topology file is shown in figure~\ref{fig:topsimple}. A parser written in C++ is developed to create the given number of nodes in ns-3 and installed the gossip protocol application on them. Thereafter, all edges are parsed and created accordingly. Each node holds an ns-3 Ipv4Interface with assigned Ipv4 address and stores the Ipv4 addresses of his neighbors.

For the simulation, we set the link rate for all connections to be 1Mbp. Also, all nodes are instructed to execute the gossip process of sending out data periodically every 5ms. The interval of requesting new data is set to 5s.

To allow the performance analysis, all nodes count the number of data packets they sent. All nodes would track how many hops the data message experienced before reaching them and they record the time when they received the data message as well. For every single simulation, we collect the information from the nodes and determine the average amount of data packets sent per node and average number of hops per node. Moreover, the information about how long it took for the message to reach the ``last'' node is also recorded.

This information is determined and stored for each of the several hundred topology files. It should be noted that due to time limitations each topology file was only simulated once. Finally, we did statistical analysis upon those collected data in the hope of verifying the assumptions we made in section~\ref{sec:problem}.


\subsection{A Subsection}


\section{Another Section}

