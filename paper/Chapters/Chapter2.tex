\chapter{Related Work}
\label{Chapter2}
\lhead{Chapter 2. \emph{Related Work}} % Write in your own chapter title to set the page header

Over the years, gossip techniques have prove to be the corner stone for building scalable and robust distributed computer network systems. This technique is often be used to design multicast protocol \cite{gossip} \cite{gupta2002efficient}, routing protocol, and broadcast protocol. Demers et. al. \cite{demers1987epidemic} demonstrated the advantages of deploying this technique in corporation for database maintenance in the early days. In recent years, this technique has been utilized in wired networks \cite{birman1999bimodal} as well as in wireless networks. Many proposed schemes whether is for wired networks or is for wireless networks, they all focused on optimizing protocol overhead, or energy consumption. The metrics they used includes network density, or nodes' energy level. 

In wired network domain, it is being used for peer-to-peer network \cite{gupta2002efficient}. In wireless network domain, it is being used for mobile ad-hoc networks \cite{vahdat2000epidemic} \cite{chandra2001anonymous}, and wireless sensor networks \cite{levis2004trickle} \cite{miller2005exploring}. There are many proposed schemes that are designed for wired network that uses network information for adaptive gossiping such as \cite{kempe2004spatial} \cite{rodrigues2003adaptive}, but those approaches are not suitable for wireless networks \cite{smart}. 

Some of the basic gossip techniques that are specifically designed for wireless networks includes fixed forward probability scheme \cite{haas2006gossip} where each node has the probability of $p$ to gossip the \msg ~to its neighbor while it has the probability of $1-p$ to not gossip the \msg ~to its neighbor. Because in this paper the authors are designing a routing protocol in a wireless ad-hoc network, each time when a node receives a new \msg, it will only pick one of its neighbors. In other words, the \emph{Fanout} here is set to be 1. The advantage of this proposed scheme is that it is easy to implement in practice. However, due to the dynamic nature of mobile ad-hoc network or wireless sensor network, a fixed forward probability can be difficult to choose. Moreover, even an optimal forward probability may become sub-optimal over time. 

To address the disadvantage of a fixed forward probability scheme, Cartigny et. al. \cite{cartigny2003border} proposed a new broadcast scheme for ad-hoc networks that would adjust \emph{\pog} based on number of neighbors a node has. The equation for calculating \emph{\pog} is $p_{gossip}=\frac{k}{n_b}$ where $k$ is the propagation factor and $n_b$ is a node's degree (number of neighbors). The minimum and maximum \emph{\pog} can be adjusted by changing propagation factor. The idea is that a node with more neighbors will have a lower probability to gossip new \msgs ~and vice versa. This scheme reduced overhead by tailoring \emph{\pog} for each node but a suitable $k$ value for various ad-hoc network topologies can still be difficult to choose. 

Another interesting proposed gossip broadcast scheme is called "Smart Gossip" \cite{smart}. It uses "family classification" method to category a node's neighbors. A node's neighbor can be in one of three categories: parent, sibling, or child. Intuitively, the more siblings a node has, the lower the \emph{\pog} will be because other siblings may have transmit the new \msg ~to the child \cite{2015survey}. Moreover, \emph{\pog} is proportional to number of children a node has \cite{2015survey}. When a node has no child, the $p_{gossip}=0$. When a node has no siblings but only child, the $p_{gossip}=1$. This advantage of this approach is that it takes nodes dependency into account while gossiping. However, this scheme can be complicated to implement and there is no update after the initial hierarchy establishment \cite{2015survey}.

In terms of reducing energy consumption while using gossip techniques, Nitnaware et. al. \cite{nitnaware2009performance} proposed a simply scheme by defining a \emph{Energy Level Threshold}. When a node's energy level drops below the threshold, this node will not gossip new \msgs it received. Otherwise, it will gossip the new \msg ~with the probability of $k$. One special case is that when a node only has one neighbor, it will gossip the new \msg ~with probability of 1 regardless of its energy level. In \cite{nitnaware2010energy}, the authors proposed to use the remaining energy fraction directly as \emph{\pog}. So the gossip probability is defined as $p_{gossip}=\frac{E_{frac}}{100}$. A node with higher remaining energy fraction will have a higher gossip probability. A more advanced energy-aware gossip based broadcast scheme is proposed by Reina etl. al. in \cite{reina2012optimization}. In this paper, the authors proposed to calculate a node's \emph{\pog} according to the following equation: $p_{gossip}=\frac{E_i - E_{min}}{E_{max}-E_{min}}$ where $E_i$ is the node's energy level, $E_{max}$ is the maximum energy level among its neighbors, and $E_{min}$ is the minimum energy level among its neighbors. This approach requires nodes to insert their energy level information when requesting for a \msg ~update. This scheme is similar comparing to "Smart Gossip" in terms of collecting neighbors information instead of focusing on the information a node itself can obtain. 

%Another energy-aware gossip broadcast protocol proposed by Machado et. al. \cite{energyMap} uses "Energy Map" to determine a node's \emph{\pog}. When a node received a new \msg, first it will check its energy level. If its energy level is below the cut-off energy, it will simply drop this \msg. Otherwise, it will calculate its \emph{\pog} based on the following equation: $p_{gossip}=\frac{d}{r}$

One thing all these paper have in common is that they all focused on adjusting \emph{\pog} to reduce protocol overhead or reduce protocol energy consumption. They never explore the possibilities of adjusting \emph{Fanout} to achieve longer network lifetime. Because some of the paper that mentioned here focused on designing a routing protocol, a \emph{Fanout} setting of 1 is the common practice. A higher \emph{Fanout} is unlikely to improve routing protocol performance and is more complicated for a protocol to maintain the routing table. 




%one paper about control gossip protocol infection pattern using adaptive fanout

%based on density, energy level, node's distance
%optimize overhead, energy consumption, 
%in wireless: for manet, wsn, iot network
%do: multicast, broadcast, routing 

%two category: fixed probability schemes, and adaptive probability schemes.
%fixed probability: 
%adaptive probability: counter-based, non-counter-based
%counter-based: density, distance, energy 
%non-counter-based: density, speed, distance, energy